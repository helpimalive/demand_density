\documentclass[APA,Times1COL]{WileyNJDv5} %STIX1COL,STIX2COL,STIXSMALL

\usepackage{threeparttable}
\usepackage{dcolumn}
\usepackage{placeins}
\usepackage{subcaption}
\usepackage{graphicx}%

\graphicspath{{../Figs/}}

\articletype{Original Article}%

\received{Date Month Year}
\revised{Date Month Year}
\accepted{Date Month Year}
\journal{Real Estate Econ.}
\volume{00}
\copyyear{2025}
\startpage{1}

\raggedbottom



\begin{document}

\title{Crowded and Expensive: Density Shift as a Measure of Demand in Large US Apartment Markets}

\author[1]{Matt Larriva, CFA}


\authormark{}
\titlemark{Density Shift as a Measure of Demand}

\address[1]{\orgdiv{Department of Real Estate Investments}, \orgname{Brookfield Asset Management}, \orgaddress{\state{New York}, \country{United States}}}



\corres{\email{matt.larriva@brookfield.com}}

\presentaddress{250 Vesey Street, New York NY 10281 }

%\fundingInfo{Text}
%\JELinfo{ejlje}

\abstract[Abstract]{We introduce the Rental Density Index (RDI)---a novel, simple, scalable measure of rental housing demand based on grouping rather than spatial density---and demonstrate its predictive and diagnostic power across U.S. metropolitan markets. Using panel data from the 100 largest MSAs (2000-20024), we show that changes in RDI reliably signal future rent growth and provide a cap over which supply leads to declining rent. Using a two-stage least squares (2SLS) framework to address simultaneity between rents and density shifts, we establish the index as a robust measure of demand. To validate its real-world utility, we conduct a suite of tests: implementing event studies of regime transitions, benchmarking performance against other forecasting models, and examine RDI-supply imbalances during supply shocks. By capturing latent demand pressures even in fully occupied markets, RDI offers a scalable, transparent alternative to traditional metrics such as occupancy or absorption. Our results suggest that RDI can serve as both a predictive tool for rent growth and a benchmark for evaluating whether supply pipelines are aligned with underlying demand.}

\keywords{density, supply and demand, multifamily rent-growth, apartment markets}

%\jnlcitation{\cname{%	
%\author{Larriva M.}
%\author{}.}
%\ctitle{Density Shift as a Measure of Demand} \cjournal{\it Real Estate Econ.}
%\cvol{2021;00(00):1--18}.}


\maketitle

\renewcommand\thefootnote{}
\footnotetext{\textbf{Abbreviations:} MSA, Metropolitan Statistical Area; RDI, relative density index; }

\renewcommand\thefootnote{\fnsymbol{footnote}}
\setcounter{footnote}{1}
\FloatBarrier
\section{Introduction}\label{sec1}
Housing shortages and affordability concerns have risen to the forefront of policy debates in major urban markets. In the United States, housing production has consistently lagged population growth for decades, contributing to an estimated national shortfall of 4.4 million housing units \cite{betancourt2022us}. Nearly half of U.S. renter households now spend over 30\% of their income on housing \cite{censusNearlyHalf}, and from 2000 to 2024, the consumer price index (CPI) for shelter exceeded the CPI for all other items by 30\% \cite{stlouisfedConsumerPrice}. At the same time, select high-growth regions have recently experienced rent declines due to a glut of new supply \cite{mott2024ThisRegion}. This juxtaposition of chronic national undersupply with localized oversupply underscores a deeper issue: the lack of a reliable metric to measure consumer housing demand.

Traditional indicators of demand in multifamily real estate, such as occupancy rates and net absorption, are informative but fundamentally supply-constrained. Occupancy rates are naturally bounded at 100\%, and absorption cannot exceed the rate of new deliveries \cite{mueller1999real}. These constraints obscure excess demand: when all available and affordable units are occupied, latent demand becomes invisible to market participants and researchers alike \cite{gabriel2001rental, sirmans1991determinants, pyhrr1999real}. This problem inhibits clear attribution of rent increases to supply versus demand dynamics \cite{pennington2021does, molloy2022housing}. Moreover, ongoing academic debate persists around whether rising rents in constrained markets result more from supply-side limitations \cite{saiz2010geographic} or from heightened demand for desirable locations \cite{davidoff2015supply}. Without a transparent, consistent demand-side metric, attempts to assess equilibrium conditions remain incomplete.

In this paper, we introduce a novel empirical measure of rental housing demand: the \textit{rental density index} (RDI), defined as the number of people per existing rental unit in a given metropolitan area. Unlike occupancy or absorption, RDI is not inherently bounded and therefore can reflect intensifying demand even in fully occupied markets. The conceptual foundation is straightforward: as rents rise, renters economize on space---whether by delaying household formation, taking on roommates, or crowding---thus increasing the ratio of people per rental unit. By observing shifts in RDI over time, we capture underlying demand pressures that traditional metrics obscure.

Present trends support this reframing of demand, as density, rental prices, and rental preferences have all shifted meaningfully. For most of its history, the U.S. was less densely populated than its high-income peers, but in 1992 this changed. In that year, the US grew to 28 people per square kilometer, outpacing the High Income countries whose average was 23 people per square kilometer \cite{worldbankPopDensity}. The most recent figures report the US with a density of 36 and the High Income nations with an average density of 27.  Over this same period, rental prices have outpaced both wages and non-shelter inflation \cite{feiveson2024rent, stlouisfedConsumerPrice}, with younger cohorts increasingly preferring rental housing \cite{fanniemaeConsumersFeeling}.

Prior literature in urban economics and real estate has studied density extensively, often linking increased geographic density to higher wages, rents, and productivity due to agglomeration effects \cite{titman2024city, liu2018vertical}. However, these studies generally define density in terms of spatial density: people per square mile or  apartments per square mile. Our focus differs: we define density as the quotient of population over occupied rental units, which provides a clearer window into the number of people actively competing for housing. Unlike geographic density, this measure is directly responsive to shifts in demand, especially during periods of supply constraint or shocks. 

Our framework builds on the standard assumption that, all else equal, households derive higher utility from larger living spaces \cite{muth1969cities,molloy2022housing}. In high-rent markets, however, the extra price per square foot may eventually exceed the marginal benefit of additional space. To economize, renters may respond by sharing units (adding roommates), downsizing to smaller apartments, or relocating to less expensive regions. Conversely, when real rents fall or new supply comes online, the marginal cost of extra space might drop below its marginal utility, enticing households to spread out, move to larger units, or reduce unit‐share.The aggregate effect of these choices combined with the new supply additions or subtractions determines the direction the Rental Density Index change. Because the RDI directly captures these space‐consumption margins, year‐over‐year changes in RDI proxy for shifts in aggregate housing demand at prevailing prices.

\begin{figure}[htb!]
	\centerline{\includegraphics[height=20pc]{rdi_rent_growth_2024.pdf}}
	\caption{Relationship between average annual change in Rental Density Index (horizontal) and average annual relative real rent growth (vertical) in the 100 largest MSAs for each year between 2001 and 2024. \label{fig:rdi_national}}
\end{figure}

Focusing on a panel of the 100 largest metropolitan statistical areas (MSA) between 2000 and 2024, we calculate the RDI. We then examine the year-over-year change in each MSA, in each year. A positive change in RDI (densification) means that population has grown faster than inventory; conversely a negative change in RDI (de-densification) corresponds to inventory growing faster than population. While this ignores the number of households and the percentage of owners versus renters, the signal from the RDI change is robust in spite of this.

This empirical strategy reveals consistent rent growth differences across multiple time horizons. Over the one-year horizon, top RDI growth  markets significantly outperform the bottom RDI growth quintile markets. The densifying markets outperform the de-densifying markets by over 100 basis points in real relative rent growth. The RDI growth also has very strong foresight with the trailing 10 year changes highly predictive of the next 10 years of supply growth and rent growth. The RDI is especially powerful when analyzed with observed supply growth. It serves as an indicator of when supply will be accretive to rent versus dillutive. In years and markets where the next year's supply exceeds the prior year's RDI growth, real rent growth is significantly below zero. Conversely, when supply is less than the RDI growth, the real rent is significantly greater than zero. These findings suggest that density-based classifications have predictive power and reflect latent demand more accurately than traditional indicators alone.

Our paper makes two principal contributions. First, we introduce the Rental Density Index (RDI), a novel, supply-unbounded metric of housing demand that can be computed at scale from readily available population and unit‐stock data. Second, we demonstrate RDI’s empirical value using a suite of use-cases applicable to investors, renters, developers, and policy-makers.

These findings carry immediate applications for all housing‐market stakeholders. For policymakers, RDI functions as an early-warning gauge of emerging shortages, enabling targeted zoning reforms, calibrated subsidy programs, or expedited permitting in precisely those submarkets under the greatest pressure—and subsequently measuring the efficacy of those measures. Institutional investors and residential developers can embed RDI trends into feasibility and risk models, avoiding the twin hazards of overbuilding in cooling metros or underbuilding in tightening ones. Finally, RDI sharpens the housing‐affordability debate by identifying the locales where rent increases are most likely to outstrip income growth, thereby guiding tenant‐protection policies, rent‐assistance programs, and other affordability interventions to the communities that need them most.

In the next section we discuss research on other measures of demand before presenting details on our proposed variable. The Data and Descriptive Statistics section describes and analyzes the density data we used, while the Empirical Analysis section presents evidence and illustrates statistical tests performed to evaluate the validity of the classifications. The final sections discuss the results of our tests before concluding with implications and further research suggestions. 

\section{Background and Literature Review}\label{sec2}

The multifamily housing market has long relied on a narrow set of demand indicators, notably occupancy rates and net absorption. These indicators, while intuitively appealing and widely used by practitioners, are fundamentally constrained by the available stock of housing units. Occupancy rates cannot exceed 100\%, and net absorption can never exceed new supply \cite{mueller1999real, gabriel2001rental}. Consequently, they fail to capture periods of latent demand where tenants cohabitate. \cite{sirmans1991determinants, pyhrr1999real}. Occupancy in particular has become an especially measure of demand since the advent of algorithmic pricing systems. So-called revenue management systems adjust rent pricing with the goal of keeping occupancy very near 96\%, \cite{calder2024coordinated} the result of which is a variable with little signal. 

This measurement limitation is consequential. Numerous empirical studies find that real rent growth often occurs in periods of high occupancy, yet they typically ascribe this to supply constraints rather than to excess demand \cite{goodman1992rental, wheaton1991realestate}. While such studies validate the predictive value of these metrics, their theoretical bounds limit their explanatory reach, particularly when trying to assess equilibrium or derive true demand elasticities \cite{pennington2021does, molloy2022housing}.

Alternative approaches to measuring demand have included econometric estimates of demand elasticity \cite{green2002measuring}, consumer preference surveys \cite{malpezzi1996rent}, and utility-based choice models \cite{rosenthal1997housing}. However, these methods either lack the spatial and temporal granularity needed for policy or investment use, or they are not publicly available in standardized formats.

The broader urban economics literature has focused on density as a related but distinct construct. Seminal work by \cite{glaeser2001cities} and \cite{duranton2004micro} positions geographic density---typically measured as people or housing units per square mile---as a proxy for agglomeration benefits. These studies find that higher density correlates with increased productivity, innovation, and wages. However, they stop short of using density as a direct measure of housing demand.

More recent research has explored density in the context of housing affordability. \cite{ahlfeldt2019economic} and \cite{albouy2015driving} argue that densification can both alleviate and exacerbate affordability issues, depending on its implementation. For instance, densification may increase supply and reduce rents in the long term but may also create localized price pressures or quality-of-life tradeoffs that drive demand elsewhere.

Our work contributes to this literature by redefining density as \textit{people per rental unit}, rather than per geographic area. We call this the Rental Density Index (RDI). This formulation allows demand to exceed supply in a measurable way: if population grows faster than units, density rises. If renters prefer space and are observed to cohabitate only at higher prices, then shifts in RDI reveal the slope of the underlying demand curve \cite{muth1969cities, molloy2022housing}.

Unlike geographic density, which may be influenced by zoning and land use policy, RDI responds directly to demographic pressures and consumer decisions. Its changes over time---\( \Delta \text{RDI} \)---can be interpreted as demand shocks, analogous to shifts in labor force participation or consumption behavior in macroeconomic models. By linking \( \Delta \text{RDI} \) to rent outcomes, we recover a market-clearing framework that identifies over- and under-supplied markets and projects likely rent growth trajectories.

Prior research has extensively modeled housing demand through price elasticity estimates \cite{green}, utility-based choice models \cite{rosenthal}, and discrete consumer preference surveys \cite{malpezzi}. These approaches contribute valuable structural insight into renter behavior, but they typically require extensive microdata on incomes, preferences, or migration patterns, and often assume equilibrium conditions. In contrast, our Rental Density Index (RDI) framework offers a simple, scalable alternative that avoids these data demands while directly capturing revealed behavior at the aggregate level. Rather than estimating the slope of the demand curve through price responsiveness alone, we observe actual changes in space consumption—crowding or de-crowding—as rents rise or fall. This allows the RDI approach to reflect latent or excess demand even in fully occupied markets, where traditional elasticity estimates may fail to detect ongoing pressures. Our contribution is thus complementary to elasticity studies: whereas elasticity models estimate how much demand shifts in response to price changes under assumed conditions, RDI directly observes whether demand pressure exists at current prices by measuring household adjustments in space per person. In doing so, we provide a practical tool for market segmentation and forecasting that operates even when underlying micro-preference data are unavailable or incomplete.

This approach also aligns with emerging calls in the real estate literature to develop forward-looking, demand-side indicators that complement the traditional focus on supply elasticity \cite{glaeser2019rethinking}. In contrast to existing studies, our method uses widely available data---population, rental unit inventory, and rent---to produce a scalable, repeatable measure of housing demand at the metro level.

In sum, our contribution lies in reinterpreting a well-studied spatial metric---density---through the lens of consumer housing decisions. By focusing on the population-to-unit ratio rather than geographic dispersion, we provide a new empirical tool to better understand and forecast real estate market dynamics.

\section{Data and Descriptive Statistics}\label{sec3}
Our data are sourced primarily from Costar and supplemented with inflation measures from the U.S. Bureau of Labor Statistics. We restrict our analysis to the 100 largest U.S. metropolitan statistical areas (MSAs) by multifamily inventory as of 2001, ensuring robust sample representation and consistency over time.

To normalize pricing data across time, we convert all nominal rent figures into real terms using the Consumer Price Index for All Urban Consumers (CPI-U), all-items, provided by the BLS. Although MSA-level inflation indices excluding rent would be ideal, such data are unavailable at a consistent and granular level for our study period. Therefore, real effective rent per square foot and rent growth metrics are adjusted using national CPI. 

We convert the rent growth figures to real as well, by subtracting the annual growth in CPI. Finally, we standardize the figures by subtracting the year's median rent growth. We do this to remove the impact of national macroeconomic shocks (i.e., COVID-19). This also serves to stationarize the data and remove the autoregressive tendencies of timeseries data.

Our novel demand variable, the Rental Density Index (RDI), is calculated by dividing population by total rental units at the MSA level. We compute its year-over-year percentage change---\(\Delta\text{RDI}\)---to reflect demand dynamics more accurately. This measure avoids the bounded structure of traditional demand indicators like occupancy and absorption, which are upper bounded by 100\%.

Other variables include supply growth, computed as the ratio of delivered units to the previous year's inventory, and lagged variables prior-year rent growth. 

We exclude New Orleans, LA from our analysis because of its dynamics around the unfortunate aftermath of Hurricane Katrina. Within two years, this MSA had the greatest and least values of RDI growth as well as the greatest and least values of real rent change. 

\begin{table*}[hbt]%
\centering
\caption{Key variables used in the empirical analysis.\label{tab:variables}}%
\begin{tabular*}{\textwidth}{@{\extracolsep\fill}ll@{\extracolsep\fill}}%
\toprule
\textbf{Variable} & \textbf{Description} \\
\midrule
\texttt{CPI}					& Consumer Price Index for All Urban Consumers: All Items in U.S. City Average \\
\texttt{real\_rentpsf}               & Costar's Rent Per Square Foot divided by indexed \texttt{CPI}\\
\texttt{real\_rent\_growth}          & Costar's Effective Rent Growth 12M less Annual percent change in \texttt{CPI} \\
\texttt{relative\_real\_rent\_growth}          & \texttt{real\_rent\_growth} less that year's median \texttt{real\_rent\_growth}\\
\texttt{inventory}             & Existing multifamily rental stock \\
\texttt{delivered}             & Units delivered in the current year \\
\texttt{pop}                   & Total MSA population (Costar/Moody’s) \\
\texttt{RDI}			       & Population divided by rental inventory \\
\texttt{delta\_RDI}	           & Year‐over‐year percentage change in \texttt{RDI} \\
\texttt{supply\_growth}        & Delivered units as \% of prior year's \texttt{inventory}\\
\bottomrule
\end{tabular*}
\end{table*}


\subsection{Variable Construction}
The key outcome variable in this study is the \textit{Rental Density Index} (RDI), calculated as the total population divided by the number of rental housing units in a given MSA-year:

\begin{equation*}
	\text{RDI}_{it} = \frac{\text{Population}_{it}}{\text{Rental Units}_{it}}.
\end{equation*}
	
Within the 100 MSAs in the study,this metric ranges from 6.8 to 64.8. It reflects the number of people per rental unit. 
\begin{figure*}[hbt!]
	\centering
	
	\begin{subfigure}[b]{0.32\textwidth}
		\includegraphics[width=\linewidth]{us.png}
		\caption{United States}\label{fig:us_choropleth}
	\end{subfigure}\hfill
	\begin{subfigure}[b]{0.32\textwidth}
		\includegraphics[width=\linewidth]{tristate.png}
		\caption{Tri-State area}\label{fig:tristate_choropleth}
	\end{subfigure}\hfill
	\begin{subfigure}[b]{0.32\textwidth}
		\includegraphics[width=\linewidth]{florida.png}
		\caption{Florida}\label{fig:florida_choropleth}
	\end{subfigure}
	
	\caption{Rental-Density-Index (RDI) choropleths at three spatial scales.}
	\label{fig:choropleth_panel}
\end{figure*}


While the RDI itself is useful for identifying how tight a market is at a given point in time, its absolute level is shaped by long-run demographic and structural trends such as rentership rates and changes in household formation. Accordingly, we focus on the \textit{year-over-year change in RDI}:

\begin{equation*}
	\Delta \text{RDI}_{it} = \text{RDI}_{it} - \text{RDI}_{it-1}.
\end{equation*}


The change in RDI (\( \Delta \text{RDI} \)) offers several advantages. First, it mitigates issues of nonstationarity in the level RDI, enabling cross-market comparison. Second, it reduces the impact of varying rentership percentages across cities. Most importantly, \( \Delta \text{RDI} \) captures demand pressure on housing stock: when it rises, it indicates more people are consolidating into fewer rental units, signaling tightening demand. When it falls, it implies renters are spreading out and absorbing more space, suggesting slack.

Theoretically, under the assumption that renters prefer more space to less, the change in RDI reveals when prices are sufficiently high to induce cohabitation and when price relief allows individuals to live separately. Thus, \( \Delta \text{RDI} \), especially when combined with rent growth, helps identify periods of excess demand or slack and allows us to trace out implied demand curves. As we show later, this approach allows us to locate the intersection between demand and supply curves using observable outcomes.


\subsection{Summary Statistics}
Table \ref{tab:summary_stats} provides descriptive statistics for the key variables across the full panel. On average, the RDI across all MSAs and years is approximately 2.31 persons per rental unit, with a standard deviation of 0.35. The average real rent per square foot is \$1.22, and the average annual growth in rental supply is 1.9\%. The missing records exist only in the beginning or ending years as the prior period and next-period values do not exist. 

\begin{table}[hbt!]
	\centering
	\caption{Summary Statistics (2000--2024)}\label{tab:summary_stats}
	\begin{tabular}{lrrrrrrrr}
	\toprule
	Variable & Mean       & Std.\ Dev.\   & Min       & 25th Pct.  & Median     & 75th Pct.  & Max        & Missing \\
	\midrule
	Population          & 2,012,976  & 2,169,231  &   175,653  &   758,060  & 1,224,731  & 2,422,333  & 14,849,020 &   0  \\
	Inventory           &   133,435  &   190,354  &    20,783  &    35,735  &    66,261  &   153,774  &  1,572,425 &   0  \\
	Supply growth       &      0.0169&      0.0152&     –0.0173&     0.0057 &    –0.0059 &     0.0239 &     0.1061 & 100  \\
	RDI                 &     18.259 &      6.852 &      6.841 &    13.737  &    17.067  &    20.967  &    64.795  &   0  \\
	RDI growth          &     –0.0078&      0.0151&     –0.1895&    –0.0163 &    –0.0059 &     0.0022 &     0.0512 & 100  \\
	Real rent (\$/sqft) &      0.9489&      0.3551 &      0.5584&     0.7164 &     0.8373 &     1.0436 &     2.9110 &   0  \\
	Real rent growth    &     –0.0035&      0.0300 &     –0.1915&    –0.0215 &    –0.0046 &     0.0114 &     0.2809 & 100  \\
	\bottomrule
\end{tabular}
\end{table}

\subsection{Coverage and Representativeness}
The sample includes approximately 1,300 MSA-year observations, covering a mix of large coastal cities, fast-growing Sun Belt metros, and slower-growing Midwestern regions. Together, these MSAs account for over 16 million of the 22 million U.S. apartment units, providing a representative snapshot of national rental dynamics. 

\subsection{Preliminary Observations}

Figure \ref{fig:sidebyside} traces the national Rental Density Index (RDI) from 2000 to 2024. Contrary to conventional wisdom about an acute housing shortage, we observe a steady decline in people per unit (from roughly 16 people-per-unit in 2000 to 13 people-per-unit in 2024). A companion histogram of ΔRDI is stationary but skewed slightly negative, confirming that most years see more units added per new person than vice versa.

\begin{figure}[!htb]
	\centering
	\begin{subfigure}[b]{0.48\textwidth}
		\centering
		\includegraphics[width=\textwidth]{rdi_growth_histogram.pdf}
		\caption{A histogram of the annual change in RDI across 100 MSAs in the years 2001-2023. The outlier to the left is New  Orleans in 2004, due to Huricane Katrina.\label{fig:rdi_hist}}
	\end{subfigure}
	\hfill
	\begin{subfigure}[b]{0.48\textwidth}
		\centering
		\includegraphics[width=\textwidth]{rdi_trends_selected_msas.pdf}
		\caption{A line graph showing the RDI values (population divided by apartment units) for the MSAs that were the min, max, and median in year 2001}
		\label{fig:rdi_lines}
	\end{subfigure}
	\caption{Histogram of Change in RDI, and Line Graph of RDI}
	\label{fig:sidebyside}
\end{figure}
This pattern poses a puzzle: if supply growth has outstripped population growth, why is everyone still worried about shortages? Occupancy simply tells you \textit{that} units are full--it cannot distinguish whether they are full by choice (everyone prefers roommates) or by necessity (everyone must share).

By contrast, year-over-year changes in RDI isolate the relative pace of new households versus new units. When rents rise faster than new construction can absorb, ΔRDI spikes, revealing latent demand pressure. When deliveries overwhelm demand, ΔRDI falls, signaling surplus. In the next section we exploit that property, regressing ΔRDI on rent and supply growth to recover the underlying demand curve.

\section{Empirical Strategy and Econometric Results}

\subsection{Identification Strategy}
We aim to estimate the causal effect of crowding---as measured by growth in the Rental Density Index (RDI)---on future rent growth. However, RDI may be endogenous to unobserved demand shocks, reverse causality, or simultaneity with rents. For example, rising rents may force household compression, inflating RDI, or latent demand factors may affect both rents and occupancy intensity.

To address this, we use a two-stage least squares (2SLS) estimation strategy, instrumenting RDI growth with an exogenous migration shock. This shock is defined as a binary indicator equal to one when international in-migration exceeds the 90th percentile across all MSA-years from 2000 to 2024. This threshold identifies years of unusually high foreign population inflows plausibly unrelated to local rent conditions. We control for population growth and sales volume growth, and include MSA and year fixed effects.

\begin{align*}
	\text{Stage 1:}\quad & \Delta \text{RDI}_{i,t} = \gamma_0 + \gamma_1 \cdot \text{Shock}_{i,t} + \gamma_2 \cdot \textbf{X}_{i,t} + \mu_i + \delta_t + u_{i,t} \\\\
	\text{Stage 2:}\quad & \text{RentGrowth}_{i,t+1} = \alpha_0 + \beta \cdot \widehat{\Delta \text{RDI}}_{i,t} + \alpha_1 \cdot \textbf{X}_{i,t} + \mu_i + \delta_t + \varepsilon_{i,t}
\end{align*}

Where:
\begin{itemize}
	\item $\text{RentGrowth}{i,t+1}$ is next-year relative real rent growth in MSA $i$,
	\item $\Delta \text{RDI}{i,t}$ is the change in Rental Density Index,
	\item $\text{Shock}{i,t}$ is the migration shock instrument (defined as above),
	\item $\mathbf{X}{i,t}$ includes controls: population growth and sales volume growth,
	\item $\mu_i$ and $\delta_t$ are MSA and year fixed effects.
\end{itemize}

\subsection{Baseline Results}
Table~\ref{tab:main-results} shows that instrumented RDI growth significantly predicts future rent growth. A one-point increase in RDI growth is associated with a 1.29 percentage point increase in next-year rent growth.

\begin{table}[h]
	\centering
	\caption{2SLS Estimates: RDI Growth and Next-Year Rent Growth}
	\label{tab:main-results}
	\begin{tabular}{lcccc} \toprule
		& Coefficient & Std. Error & t-stat & p-value \\ \midrule
		RDI\textbackslash growth & 1.2865 & 0.4171 & 3.085 & 0.0020 \\
		Population Growth & -0.0038 & 0.0019 & -1.963 & 0.0496 \\
		Sales Volume Growth & 4.1653 & 0.8401 & 4.958 & 0.0000 \\
		\midrule
		MSA Fixed Effects & Yes & & & \\
		Year Fixed Effects & Yes & & & \\
		First-stage F-stat & 45.051 & & &\\
		Observations & 1,863 & & & \\ \bottomrule
	\end{tabular}
\end{table}

\subsection{Robustness Tests}
We conduct two robustness tests to validate our identification strategy.

\textbf{1. Placebo Test.} Using same-year rent growth as the dependent variable weakens the effect of RDI growth, suggesting RDI leads rent rather than the reverse.

\begin{table}[h]
	\centering
	\caption{Placebo Test: RDI Growth on Same-Year Rent Growth}
	\label{tab:placebo}
	\begin{tabular}{lcccc} \toprule
		& Coefficient & Std. Error & t-stat & p-value \\ \midrule
		RDI\textbackslash growth & 1.0994 & 0.4126 & 2.665 & 0.0077 \\
		Population Growth & -0.0011 & 0.0019 & -0.573 & 0.567 \\
		Sales Volume Growth & 5.5281 & 0.8555 & 6.462 & 0.0000 \\
		\midrule
		First-stage F-stat & 76.214 & & & \\
		Observations & 1,863 & & & \\ \bottomrule
	\end{tabular}
\end{table}

\textbf{2. Lagged Instrument.} Replacing the contemporaneous migration shock with its lag ($\text{Shock}_{i,t-1}$) yields a similar point estimate with slightly lower precision.

\begin{table}[h]
	\centering
	\caption{Lagged Instrument: RDI Growth and Next-Year Rent Growth}
	\label{tab:lagged}
	\begin{tabular}{lcccc} \toprule
		& Coefficient & Std. Error & t-stat & p-value \\ \midrule
		RDI\textbackslash growth & 1.2803 & 0.7152 & 1.790 & 0.0734 \\
		Population Growth & -0.0037 & 0.0033 & -1.100 & 0.2711 \\
		Sales Volume Growth & 4.5286 & 0.8548 & 5.298 & 0.0000 \\
		\midrule
		First-stage F-stat & 40.087 & & & \\
		Observations & 1,782 & & & \\ \bottomrule
	\end{tabular}
\end{table}

\subsection{Interpretation and Limitations}
Across all specifications, RDI growth remains a positive and statistically significant predictor of next-year rent growth. The placebo test demonstrates that the effect weakens for contemporaneous rents, supporting a forward-looking relationship. The lagged-instrument model confirms the robustness of this effect, though with expected attenuation.

These results support the interpretation that crowding---as measured by RDI---reflects latent demand pressure that translates into future rent growth. Our findings are consistent with a causal interpretation and are robust to multiple instrument timing and exclusion validity checks. Results are stable across alternative shock thresholds, including the 85th and 95th percentiles (see Appendix Table A1).

\section{Forecasting Validation and Real-World Performance}

\subsection{Predictive Segment Testing: ANOVA on RDI Growth}
We test whether observed changes in RDI can segment markets into meaningful groups in terms of future rent performance. Specifically, we split the samples into two regimes: those with positive $\Delta$RDI and those negative $\Delta$RDI. We then examine whether average next-year relative real rent growth differs significantly across these groups.

Table~\ref{tab:anova-results} shows that markets with positive RDI growth experience average next-year rent growth of 41 basis points above the mean rent growth (significant at p$\leq 3.14^{-14}$), while markets with non-positive RDI growth exhibit an average decline of 15 basis points (significant at p$\leq$0.0026). The difference is statistically significant, as confirmed by both a one-way ANOVA and Tukey’s HSD post-hoc test.

\begin{table}[h]
	\centering
	\caption{ANOVA of Next Year's Relative Real Rent Growth Grouped by RDI}
	\label{tab:anova-results}
	\begin{tabular}{lcccccc} \toprule
		$\Delta$RDI Group & Mean (bps) & SE (bps) & 95\% CI Lower & 95\% CI Upper & $n$ & $p$-value \\ \midrule
		RDI Decline & -15 & 5 & -25 & -5 & 1538 & 0.0026 \\
		RDI Growth & 41 & 7 & 31 & 52 & 762 & $3.14^{-14}$ \\
		\bottomrule
	\end{tabular}
\end{table}

A two-way ANOVA confirms the significance of the group difference ($F = 49.2$, $p \leq 2.98 \times 10^{-12}$), and a Tukey HSD test further validates that the difference in means is statistically distinguishable.
\begin{table}[h]
	\centering
	\caption{Two-Way ANOVA Test Results}
	\label{tab:two-way-anova}
	\begin{tabular}{lcccc} \toprule
		Term & Sum Sq & DF & F & p-value \\ \midrule
		C(demand) & 0.0165 & 1.0 & 49.23 &  $2.98^{-12}$\\
		Residual & 0.7699 & 2298.0 & --- & --- \\
		\bottomrule
	\end{tabular}
\end{table}

\begin{table}[h]
	\centering
	\caption{Tukey HSD Test Results}
	\label{tab:tukey}
	
	\begin{tabular}{lcccccc} \toprule
		Group 1 & Group 2 & Mean Diff & p-adj & Lower & Upper & Reject \\ \midrule
		$\Delta$ RDI $\leq$0 & $\Delta$RDI>0 & 0.0057 & 0.0000 & 0.0041 & 0.0073 & True \\
		\bottomrule
	\end{tabular}
\end{table}

These results show that even without instrumentation, RDI growth serves as a powerful signal for forward rent performance. Practitioners can use RDI segmentation as a lightweight, interpretable classification rule for identifying tight rental markets.

\subsection{Event Study of RDI Regime Transitions}
We next evaluate how rent growth responds dynamically to transitions into and out of RDI-growth regimes. Specifically, we examine rent growth before and after a market switches into a state of positive  $\Delta$RDI (increasing crowding) or  negative  $\Delta$RDI (declining crowding). We continue to examine results in terms of real rent growth relative to the mean rent growth of that year, to avoid capturing rent changes due to macro conditions.

Figure~\ref{fig:event-study} illustrates rent growth over a five-year window centered on the regime switch. The blue line represents transitions into crowding markets, the red line transitions into de-densifying markets, and the gray line shows cases where RDI status does not change.

\begin{figure}[h]
	\centering
	\includegraphics[width=0.8\textwidth]{event_study.pdf}
	\caption{Real Relative Rent Growth Before and After a Market Switches From/To Positive/Negative $\Delta$RDI}
	\label{fig:event-study}
\end{figure}

Table~\ref{tab:event-means} reports differences in average rent growth before and after the regime shift. Markets transitioning into crowded segments see a 48 basis point increase in rent growth, while those exiting see a 29 basis point decline. No significant change is observed when RDI status remains unchanged, as expected.

\begin{table}[h]
	\centering
	\caption{Mean Rent Growth Before and After RDI Regime Transition}
	\label{tab:event-means}
	\begin{tabular}{lcccc} \toprule
		Transition Type & Mean Before & Mean After & Difference & p-value\\ \midrule
		Switched to $\Delta \text{RDI}>0$ & -31 & 17 & 48 & 0.0027\\
		Switched to $\Delta \text{RDI}\leq0$ & 52 & 23 & -29 & 0.0089\\
		No change & -3 & -8 & -5 & 0.6336\\
		\bottomrule
	\end{tabular}
\end{table}

These dynamics confirm that the onset of crowding pressures, as measured by RDI, precedes statistically and economically meaningful changes in rent performance. RDI transitions can therefore serve as timely indicators of shifts in market pricing power.


\subsection{Forecast Spread Comparison: RDI vs ARIMA vs Naïve}
We compare the predictive strength of RDI growth against ARIMA-based and naive trailing average models in forecasting real rent growth across multiple time horizons. For each method, we rank MSAs into quintiles based on their predicted growth and compute the realized top-minus-bottom quintile spread.
Figure~\ref{fig:spread-comparison-quadrant} displays forecast accuracy over one-year, three-year, five-year, and ten-year windows. The RDI-based approach consistently delivers stronger spread segmentation, particularly at longer horizons where traditional time-series methods tend to degrade.

\begin{figure}[h!]
	\centering
	\begin{tabular}{cc}
		\includegraphics[width=0.45\textwidth]{spread_comparison_over_time_1yr.pdf} &
		\includegraphics[width=0.45\textwidth]{spread_comparison_over_time_3yr.pdf} \\
		\textbf{(a)} 1-Year Forecast & \textbf{(b)} 3-Year Forecast \\
		\includegraphics[width=0.45\textwidth]{spread_comparison_over_time_5yr.pdf} &
		\includegraphics[width=0.45\textwidth]{spread_comparison_over_time_10yr.pdf} \\
		\textbf{(c)} 5-Year Forecast & \textbf{(d)} 10-Year Forecast \\
	\end{tabular}
	\caption{Top-minus-Bottom Quintile Rent Growth Spread by Forecast Method and Horizon}
	\label{fig:spread-comparison-quadrant}
\end{figure}

These results reinforce the value of RDI as not only a causal explanatory variable but also a forward-looking predictive signal. Forecasting long-term rent growth is notoriously difficult; most models lose power beyond a few years. The fact that the RDI-based approach maintains signal strength across one-, three-, five-, and ten-year periods underscores its robustness. While time series models rely on trailing trends, RDI captures forward-looking, structural occupancy pressure—making it an effective long-horizon indicator of underlying demand.






\section{Discussion}

Our findings reinforce the utility of rental density---measured as population per occupied rental unit---as a robust proxy for demand in multifamily housing markets. Traditional demand-side variables such as occupancy and absorption are bounded above by 100\%, rendering them structurally incapable of expressing excess demand. By contrast, our proposed measure, the change in Rental Density Index (\(\Delta\text{RDI}\)), accommodates marginal and nonlinear shifts in tenant behavior, particularly around unit sharing or household formation.

The empirical evidence supports this theoretical proposition. When supply and demand curves were derived from historical \(\Delta\text{RDI}\) and supply growth data, we observed that the relative position of actual market conditions to the implied equilibrium---quantified as overpriced/underpriced and oversupplied/undersupplied---was significantly associated with next-year rent growth. Specifically, markets that were both overpriced and undersupplied experienced statistically higher rent growth than other combinations. Conversely, renter-favorable markets (underpriced and oversupplied) exhibited lower growth or even declines. 

This segmentation has both explanatory and predictive power. Our simplified ANOVA model revealed statistically significant differences in next-year rent growth across these market states. We presented an event-study model showing the impact of being classified as 'Renter-Friendly' and 'Landlord-Friendly' in the years following the classification. Finally we compared the predictive power of our classification system to that of ARIMA models and naive models used to forecast market rent-growth. In this comparison as well, our model outperformed. 

While the number of MSAs in each state was not evenly distributed across years, this asymmetry reflects genuine market dynamics rather than bias or misclassification. For example, in periods of expansion, we naturally observe more overpriced or undersupplied markets, consistent with cyclical pressures on housing. Nonetheless, even in years with few markets in a given segment, those groupings consistently displayed expected behavior in rent trajectories.

These findings reinforce the value of \(\Delta\text{RDI}\) as a continuous, forward-looking measure of housing demand, especially when contrasted with backward-looking or constrained alternatives. They also suggest that equilibrium misalignment---in both price and quantity---can be quantified in a way that is both actionable for investors and meaningful for policymakers.

\section{Conclusion}

This paper contributes to the literature on housing market dynamics by introducing the change in Rental Density Index (\(\Delta\text{RDI}\)) as a scalable, interpretable measure of demand. Unlike occupancy or absorption, \(\Delta\text{RDI}\) captures variation in consumer preference through household formation behavior---a critical but often overlooked channel of adjustment in rental markets.

By modeling supply and demand curves separately and estimating their intersection, we were able to derive implied market-clearing quantities and prices for each metropolitan area in our dataset. These derived values allowed for the segmentation of MSAs into four quadrants based on price and supply misalignment. Across the 100 largest U.S. markets from 2010 to 2023, these quadrants were statistically associated with next-year rent growth in ways consistent with theory: undersupplied and overpriced markets performed best; oversupplied and underpriced markets performed worst.

This demand proxy also showed moderate predictive power when used in a linear forecasting framework, suggesting that it may serve as a leading indicator of rent pressure---particularly when combined with supply metrics. The ability to anticipate future rent dynamics is valuable to developers, institutional landlords, and policymakers seeking to manage affordability and stability in rental housing.

Future work should explore refinement of \(\Delta\text{RDI}\) to account for compositional changes in renter populations and household structures. Additional improvements might include the integration of MSA-specific CPI indices or disaggregated demand inputs, such as migration, income distributions, or age cohorts. Nonetheless, the present analysis establishes \(\Delta\text{RDI}\) as a conceptually valid and empirically useful measure of multifamily housing demand, with applications in forecasting, pricing, and supply planning.


%\backmatter
\bmsection*{Author contributions}

All authors contributed equally

\bmsection*{Acknowledgments}


\bmsection*{Financial disclosure}

\bmsection*{Conflict of interest}


\bmsection*{Data Disclosure}


\bibliography{wileyNJD-APA}
\bmsection*{Supporting information}

Additional supporting information may be found in the
online version of the article at the publisher’s website.










\nocite{*}% Show all bib entries - both cited and uncited; comment this line to view only cited bib entries;


\end{document}
